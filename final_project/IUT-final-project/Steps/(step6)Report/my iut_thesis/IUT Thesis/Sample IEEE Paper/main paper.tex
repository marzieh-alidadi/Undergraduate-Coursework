\documentclass[A4paper,conference]{IEEEtran}

\usepackage{amsmath,amssymb}

\IEEEoverridecommandlockoutsz


\title{Parity Space Residual Generator for \\ Non-uniformly Sampled Systems: Direct Designs}
\date{}
\author{Iman Izadi, Sirish L. Shah and Tongwen Chen
\thanks{I. Izadi is with the Department of Electrical \& Computer Engineering,
Isfahan University of Technology.}
\thanks{S.L. Shah is with the Department of Chemical \& Materials Engineering,
University of Alberta.}
\thanks{T. Chen is with the Department of Electrical \& Computer Engineering,
University of Alberta.}}

%*********************************************************************
\begin{document}
\maketitle

%************** Abstract and Keywords *********************
\begin{abstract}

Non-uniform sampled-data systems are widely found in industry. In
these systems the process output is sampled and the control input is
generated at non-uniformly distributed time instants. In this paper,
an optimal parity based residual generator is developed for
non-uniform sampled-data systems. In the direct approach used here,
the intersample behavior of fault and unknown inputs is captured by
introducing operators that map continuous-time signals to
discrete-time signals.

\end{abstract}

\begin{keywords}
Fault detection; Sampled-data systems; Non-uniform sampling; Residual generation.
\end{keywords}
%************** Abstract and Keywords *********************

%************** Introduction *********************
\section{Introduction}\label{sec1}
Modern industrial control systems are widely exposed to faults which
can cause undesirable performance, instability, total failure of the
system and even dangerous situations. In order to maintain quality,
reliability and safety, faults should be promptly detected and
identified so that appropriate remedies can be applied. The problem
of fault detection and isolation (FDI) has widely been studied in
the past decades and numerous design methods are available in the
literature \cite{ChenPatton,Isermann}.

Traditionally a sampled-data controller/FDI is designed using
indirect approach. In the indirect approach, design is carried out
through an intermediate step of discretizing a continuous-time
system. The discretization usually involves some level of
approximation which makes the design unacceptable if sampling
intervals are large. As an alternative approach, direct methods have
been used to design sampled-data controllers and more recently
sampled-data FDI \cite{Chen,Izadi2,Zhang01a}. In direct
approach, a discrete-time controller/FDI is directly designed for
the continuous-time process taking into account the hybrid nature of
the sampled-data system. Therefore, no approximation is involved in
direct designs.

\setlength{\unitlength}{0.8mm}
\begin{figure}[tbp]
\begin{center}
\begin{picture}(90,65)(0,7)
\put(41,62){\makebox(0,0){$d(t)$}}
\put(51,62){\makebox(0,0){$f(t)$}} \put(42,60){\vector(0,-1){10}}
\put(49,60){\vector(0,-1){10}}
\put(35,38){\framebox(20,12){Process}} \put(55,44){\line(1,0){30}}
\put(70,47){\makebox(0,0){$y(t)$}} \put(85,44){\line(0,-1){17}}
\put(85,27){\vector(-1,0){10}}
\put(65,23){\framebox(10,8){\small{A/D}}}
\multiput(64,27)(-2,0){5}{\circle*{0.03}}
%\put(62,30){\makebox(0,0){$\psi(k)$}}
\put(55,27){\vector(-1,0){0}}
\put(35,22){\framebox(20,10){Controller}}
\multiput(34,27)(-2,0){5}{\circle*{0.03}}
\put(25,27){\vector(-1,0){0}}
%\put(31,30){\makebox(0,0){$\upsilon(k)$}}
\put(15,23){\framebox(10,8){\small{D/A}}} \put(5,27){\line(1,0){10}}
\put(5,27){\line(0,1){17}} \put(5,44){\vector(1,0){30}}
\put(20,47){\makebox(0,0){$u(t)$}} \put(37,7){\framebox(16,10){FDI}}
\multiput(60,27)(0,-2){9}{\circle*{0.03}}
\multiput(60,11)(-2,0){4}{\circle*{0.03}}
\put(53,11){\vector(-1,0){0}}
\multiput(31,27)(0,-2){7}{\circle*{0.03}}
\multiput(31,15)(2,0){3}{\circle*{0.03}}
\put(37,15){\vector(1,0){0}}
\multiput(36,10)(-2,0){16}{\circle*{0.03}}
\put(4,10){\vector(-1,0){0}} \put(15,14){\makebox(0,0){residual}}
\end{picture}
\end{center}
\caption{FDI in a sampled-data scheme} \label{fig1}
\end{figure}


In this paper, we develop an optimal FDI for non-uniform
sampled-data systems using direct design. The FDI which is based on
the parity space approach is designed in three steps: First the
relationship between the input and output of the process in a
certain time frame is obtained. Then a residual generator is
constructed based on the input-output relation. Finally the design
parameter is calculated by minimizing a performance index. As the
main assumption in direct approach, the fault and disturbance input
can arbitrarily vary over time. This is in contrast to the
assumption usually made in indirect approach that fault and
disturbance signals are constant over the sampling intervals. The
output sampling and control updating times can also be arbitrarily
distributed over time and there is no need for them to follow a
periodic pattern. This makes the residual generator applicable to
general non-uniform sampled-data systems.

The paper is organized as follows. In Section~\ref{sec2}, the parity
space approach of residual generation and other required
preliminaries are reviewed, and also the discretion of the system is
given. Final conclusions and remarks are
given in Section~\ref{sec9}.

%*********************************************************************
\section{Preliminaries}\label{sec2}

\subsection{Parity-space Approach}
Consider the discrete-time system
\begin{equation*}
\left \{\begin{array}{l}
 x(k+1)=Ax(k)+Bu(k)+Ed(k)+Ff(k) \\
 y(k)=Cx(k) \\ %+Du(k)+D_dd(k)+D_ff(k) \\
 \end{array}\right.
\end{equation*}
where $x(k)\in \mathbb{R}^{n_x}$ is the state vector, $u(k)\in
\mathbb{R}^{n_u}$ the vector of control input, $y(k)\in
\mathbb{R}^{n_y}$ the vector of process output, $d(k)\in
\mathbb{R}^{n_d}$ the vector of unknown inputs (e.g., disturbance,
noise, model mismatch, etc.) and $f(k)\in \mathbb{R}^{n_f}$ the
vector of faults to be detected. $A$, $B$, $C$, $E$, and $F$ are
known matrices of appropriate dimensions.

For a fixed number $s$, referred to as the order of parity relation,
define $y_s(k)$ as
\begin{equation*}
  y_s(k) = \left[
    \begin{array}{c}
    y(k-s) \\
    y(k-s+1) \\
    \vdots \\
    y(k) \\
    \end{array}
    \right]_{(s+1)n_y \times 1},
\end{equation*}
$u_s(k)$, $d_s(k)$ and $f_s(k)$ are also defined similarly. It can
be easily shown that $y_s(k)$, $u_s(k)$, $d_s(k)$ and $f_s(k)$ are
related through the following expression
\begin{equation}\label{parrel}
    y_s(k) = H_ox(k-s) + H_u u_s(k) + H_d d_s(k) + H_f f_s(k),
\end{equation}
where
\begin{equation*}
  H_{o} = \left[
    \begin{array}{c}
    C \\
    CA \\
    \vdots \\
    CA^s \\
    \end{array}
    \right],
\end{equation*}
\begin{equation*}
  H_{u} = \left[
    \begin{array}{ccccc}
      0 & 0 & \cdots & 0 & 0 \\
      CB & 0 & \cdots & 0 & 0 \\
      \vdots & \vdots &  & \vdots & \vdots \\
      CA^{s-1}B & CA^{s-2}B & \cdots & CB & 0 \\
    \end{array}
    \right].
\end{equation*}
$H_d$ and $H_f$ are defined similar to $H_u$. Based on
(\ref{parrel}), a parity space based residual generator can be
formulated as
\begin{equation*}%\label{SRres}
    r(k)=v_s\big(y_s(k)-H_{u}u_s(k)\big),
\end{equation*}
where $r(k)\in \mathbb{R}$ is the residual. 

\subsection{Operator Norm and Adjoint Operator}
\newcommand{\norm}[1]{\left\lVert#1\right \lVert}
\newcommand{\innerp}[1]{\left \langle#1\right \rangle}

Consider Hilbert spaces $\mathcal{X}$  and $\mathcal{Y}$ with inner
products $\innerp{x_1,x_2}_\mathcal{X},~x_1,x_2 \in \mathcal{X}$ and
$\innerp{y_1,y_2}_\mathcal{Y},~ y_1,y_2 \in \mathcal{Y}$,
respectively. $\mathcal{X}$ and $\mathcal{Y}$ are not necessarily
the same space, and even if they are, the inner products can be
different. The norms of members of $\mathcal{X}$ and $\mathcal{Y}$
are defined using the corresponding inner products as
$\norm{x}_\mathcal{X}^2 = \innerp{x,x}_\mathcal{X},~ x\in
\mathcal{X}$ and $\norm{y}_\mathcal{Y}^2 =
\innerp{y,y}_\mathcal{Y},~ y\in \mathcal{Y}$. Also assume that
$T:\mathcal{X} \rightarrow \mathcal{Y}$ is a bounded operator that
maps $\mathcal{X}$ to $\mathcal{Y}$. The adjoint of $T$, denoted by
$T^*$, is the unique bounded operator mapping $\mathcal{Y}$ to
$\mathcal{X}$ that satisfies
\begin{equation*}
    \innerp{Tx,y}_\mathcal{Y} = \innerp{x,T^*y}_\mathcal{X}, \quad x \in \mathcal{X},~ y
    \in \mathcal{Y}.
\end{equation*}
It can be easily shown that the adjoint of a constant matrix is its
transpose.

The induced norm of the operator $T$ is defined by
\begin{equation*}
    \norm{T} = \sup_{\norm{x}_\mathcal{X}\leq1} \norm{Tx}_\mathcal{Y}.
\end{equation*}
It is a well known fact  that
\begin{equation}\label{norms}
    \norm{T}^2 = \norm{T^*}^2 = \norm{T^*T} = \norm{TT^*}.
\end{equation}

\subsection{Process Description}
Consider an LTI, strictly proper, continuous-time process with the
following state-space representation
\begin{equation}\label{mainsys}
\left \{\begin{array}{l}
 \dot{x}(t)=Ax(t)+Bu(t)+Ed(t)+Ff(t) \\
 y(t)=Cx(t) \\
 \end{array}\right.
\end{equation}
where $x(t)\in \mathbb{R}^{n_x}$ is the state vector, $u(t)\in
\mathbb{R}^{n_u}$ the known vector of control input, $y(t)\in
\mathbb{R}^{n_y}$ the vector of process output, $d(t)\in
\mathbb{R}^{n_d}$ the vector of unknown input (to represent
disturbance, noise, model mismatch and other uncertainties) and
$f(t)\in \mathbb{R}^{n_f}$ the vector of fault to be detected. $A$,
$B$, $C$, $E$ and $F$ are known matrices of appropriate dimensions.
The assumption of strictly properness is standard in the
sampled-data literature and necessary for boundedness of the
sampling operator. In practice, because of antialiasing  filters
that are used before sampling, the systems are always strictly
proper.

In the sampled-data framework, the measured output of the
continuous-time process is sampled and transferred to a computer
using an analog-to-digital converter (A/D). The control signal is
then generated by the computer and applied to the process using a
digital-to-analog converter (D/A). In a non-uniform sampled-data
system, the output is sampled at non-uniformly spaced time instants
and/or the control input is generated at non-uniformly spaced time
instants. In this paper, for simplicity we assume that the
input/output are generated/sampled synchronously at the same times.

Let $T=\left\{t_0,t_1,t_2,\cdots\right\}$ be the set of time
instants when the output is sampled (or the input is updated). Let
$\ell_T(\mathbb{Z})$ be the vector space of all discrete-time
signals corresponding to the time instants in $T$. Notice that the
discrete-time signals in $\ell_T(\mathbb{Z})$ have no practical
meaning unless the corresponding time instants, given by $T$, are
known. Let $\mathcal{L}(\mathbb{R})$ be the vector space of all
continuous-time signals.

The D/A converter is modeled by non-uniform (zero-order) hold
operator $H_T: \ell_T(\mathbb{Z}) \rightarrow
\mathcal{L}(\mathbb{R})$ defined as
\begin{equation*}
    u(t)=H_T \upsilon_T(k)= \upsilon_T(k), \quad t_k \leq t < t_{k+1}.
\end{equation*}
Here $\upsilon_T(k)$ represent the discrete-time input.

The A/D converter is also modeled by non-uniform sampling operator
$S_T:\mathcal{L}(\mathbb{R}) \rightarrow \ell_T(\mathbb{Z})$ defined
as
\begin{equation*}
    \psi_T(k)=S_Ty(t)=y(t_k),
\end{equation*}
where $\psi_T(k)$ denotes the discrete-time output.

\section{Conclusions}\label{sec9}
In this paper, we presented a direct method to design optimal parity
based residual generator for non-uniform sampled-data systems. In
direct design, no assumption is made on fault and disturbance input
and hence there is no approximation in the solution. As a result, 
the relationship between fault/disturbance and residual is
expressed in terms of an operator rather than a matrix. However, it
was shown that the norm of the operator is equal to the norm of a
certain matrix. Therefore, the optimization problem can be converted
to a regular matrix problem whose solution is known.

%*********************************************************************
\begin{thebibliography}{99}

\bibitem{Chen}
T. Chen and B. Francis, {\em Optimal Sampled-Data Control Systems},
Springer, New York, 1995.

\bibitem{ChenPatton}
J. Chen and R. Patton, \emph{Robust Model-Based Fault Diagnosis for
Dynamic Systems}, Kluwer Academic Publishers, Boston, 1999.

\bibitem{Isermann}
R. Isermann, \emph{Fault-Diagnosis Systems: An Introduction from
Fault Detection to Fault Tolerance}, Springer, Berlin, 2006.

\bibitem{Izadi2}
I. Izadi, T. Chen and Q. Zhao, ``Norm invariant discretization for
sampled-data fault detection'', \emph{Automatica}, vol.~41,
pp.~1633--1637, 2005.

\bibitem{Zhang01a}
P. Zhang, S.X. Ding, G.Z. Wang and D.H. Zhou, ``An FDI approach for
sampled-data systems'', \emph{Proceedings of the American Control
Conference}, pp.~2702--2707, 2001.

\end{thebibliography}

\end{document}
