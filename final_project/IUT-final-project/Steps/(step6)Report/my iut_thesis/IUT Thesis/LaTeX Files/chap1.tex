% Chapter 1
\chapter{مقدمه}

اﻣﺮوزه اﯾﻨﺘﺮﻧﺖ اﺷﯿﺎء\LTRfootnote{IOT} و ﺳﯿﺴﺘﻢ‌ﻫﺎی راﯾﺎﻓﯿﺰیکی\LTRfootnote{Cyber-Physical systems}   در مسائل روزمره بسیار کاربرد دارند. از جمله ﮐﺎرﺑﺮدﻫﺎی این ﺳﯿﺴﺘﻢ‌ها در زندگی روزﻣﺮه، ﻫﻮﺷﻤﻨﺪﺳﺎزی لوازم خانگی و ساعت‌های هوشمند است. بنابراین، ﺑﺎﺗﻮﺟﻪ ﺑﻪ اﯾﻨﮑﻪ این ﺳﯿﺴﺘﻢ‌ﻫﺎ کارﻫﺎی ﺣﺴﺎسی را اﻧﺠﺎم می‌دﻫﻨﺪ، اطمینان از درستی عملکرد آن‌ها بسیار مهم است. همچنین، اطمینان از درستی عملکرد مواردی مانند چگونگی ﻣﺪﯾﺮﯾﺖ وﻗﻔﻪ‌ﻫﺎ\LTRfootnote{interrupts} ، ﻣﺤﺪودﯾﺖ ﺣﺎﻓﻈﻪ و سبک وزن ﺑﻮدن ﻋﻤﻠﯿﺎت، بسیار مهم است؛ چرا که مسائلی مانند کمبود حافظه و ﻣﻨﺎﺑﻊ در دﺳﺘﮕﺎه‌های ﮐﻢ ﻣﺼﺮف، ﺟﻨﺒﻪ‌ﻫﺎی ﻗﺎبل اﻃﻤﯿﻨﺎن در ﺳﯿﺴﺘﻢ‎ﻋﺎﻣﻞ‎های این دﺳﺘﮕﺎه ها را دچار ﭼﺎﻟﺶ می‌کند. بنابراین، لازم است عملکرد آن ها به صورت رسمی مورد تست و ارزیابی قرار گیرد\cite{7}. یکی از فرایند های حائز اهمیت در این تست و وارسی رسمی، تبدیل و ﻣﺪل‌سازی ﮐﺪ ﻣﻨﺒﻊ ﺳﯿﺴﺘﻢ ﻋﺎﻣﻞ از زبان C ﺑﻪ Promela ، به عنوان یک زبان مدل‌سازی؛ و بالعکس، تبدیل از زبان مدل‌سازی، به کد منبع سیستم عامل است. با توجه به اینکه این تبدیل، عملیاتی طاقت فرساست، به رویکردی خودکار برای انجام آن نیاز است.
\\
با توجه به اهمیت بیان شده برای مدل‌سازی و وارسی رسمی که با استفاده از زبان مدل‌سازی Promela  صورت می‌گیرد، در این پژوهش سعی شده است تا با بررسی قوانین استخراج شده از ساختار موجود در دو زبان Promela و C ، ساختار و نحوه ی تبدیل خودکار کد ﻣﻨﺒﻊ ﺳﯿﺴﺘﻢ ﻋﺎﻣﻞ از زبان C ﺑﻪ Promela ، به عنوان یک زبان مدل‌سازی ابداع شود.


\section{پیشینه تحقیق}
 
در ابتدا بررسی هایی در جهت شناسایی ابزار های\LTRfootnote{tools} موجود احتمالی که عملیات تولید مدل را به صورت خودکار انجام می‌دادند، صورت گرفت. طبق بررسی های صورت گرفته، ابزار هایی که به این منظور و برای تولید خودکار مدل در دسترس است، دارای نقاط ضعف قابل توجهی هستند. درنتیجه طی این پژوهش به بررسی این نقاط ضغف و امکان رفع آن ها پرداخته‌شد.


\section{اهداف و دستاوردهای تحقیق}
با توجه به بررسی های صورت گرفته، مشخص شد که کار بر روی ابزار های موجود و رفع نقاط ضعف و کاستی های آن ها، به صرفه تر از تولید ابزار جدید خواهدبود. بنابراین، در ادامه، به مقایسه ی عملکرد این ابزار ها و استخراج نقاط ضعف تولید کد در هر یک از آن ها و ارائه ی راه حل برای رفع کاستی های آن ها پرداخته‌شد.


\section{ساختار گزارش}
پیش از بیان شرح اصلی این پژوهش، شناخت برخی مفاهیم و ابزار ها ضرورت دارد. بنابراین، در فصل دوم به بیان برخی مفاهیم پایه، جهت کسب آگاهی نسبت به آن‌ها پرداخته‌می‌شود. در فصل سوم، برخی ابزارهای مشابه و مرتبط با این پژوهش بررسی می‌شود، و ویژگی‌ها و کاستی‌های آن‌ها بیان می‌شود. در فصل چهارم، به شرح روش پیشنهادی برای اصلاح ترتیب اجرای رویه‌ها در مدل Promela تولید شده توسط Modex ، پرداخته شده‌است. در فصل پنجم، نتیجه‌گیری از پژوهش انجام شده آورده‌شده‌است.

