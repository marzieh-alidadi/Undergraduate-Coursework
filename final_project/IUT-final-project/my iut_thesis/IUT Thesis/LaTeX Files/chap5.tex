% Chapter 5
\chapter{نتیجه‌گیری}

%%%%%%%%%%%%%%%%%%%%%%%%%%%%%%%
%%%%%%%%%%%%%%%%%%%%%%%%%%%%%%%
\section{پژوهش حاضر}
در این پژوهش سعی بر آن شد تا با تمرکز بر روی مدل‌های با فراخوانی‌های پیچیده‌ی توابع، بهبودی در عملکرد ابزار Modex برای تولید خودکار مدل Promela از کد C صورت گیرد. در این راستا، یک ماژول خودکار توسعه داده‌شد، که فایل Promela تولید شده توسط Modex را به عنوان ورودی دریافت می‎‌کند، و با اعمال اصلاحات بیان شده در فصل پیشین، مدلی بهبودیافته به عنوان خروجی تولید می‌کند.

%%%%%%%%%%%%%%%%%%%%%%%%%%%%%%%
\section{کارهای آتی}
با توجه به بررسی‌هایی که روی کاستی‌های Modex در تولید خودکار کد Promela انجام شد؛ در آینده تلاش برای تعمیم امکان فراخوانی توابع و بهبود مسیر اجرا در هنگام وجود توابع بازگشتی\LTRfootnote{recursive functions} و توابع دارای پارامتر ورودی، صورت خواهد‌گرفت. به علاوه، سعی می‌شود، مطالعات برای اصلاح تبدیل اشاره‌گرها ادامه‌یابد. همچنین ویژگی‌ها\LTRfootnote{properties} و کارایی\LTRfootnote{performance} مدل‌های حاصل از روش‌های پیشنهادی، بررسی خواهدشد.