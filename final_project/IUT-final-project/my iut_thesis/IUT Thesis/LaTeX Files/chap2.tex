% Chapter 2
\chapter{مفاهیم پایه}

%%%%%%%%%%%%%%%%%%%%%%%%%%%%%%%
%%%%%%%%%%%%%%%%%%%%%%%%%%%%%%%
\section{سیستم‌های هم‌زمان}
همزمانی\LTRfootnote{concurrency} به معنای انجام چند عملیات در یک زمان واحد است. هنگامی که در یک سیستم، چند نخ\LTRfootnote{thread} اجرایی به صورت موازی\LTRfootnote{parallel}  اجرا شوند، همزمانی رخ می‌دهد. هنگامی که همزمانی رخ می‌دهد، ممکن است برای مثال چند نخ اجرایی به صورت همزمان  به یک منبع داده در سیستم دسترسی پیدا کنند. در این صورت، امکان بروز یک سری مشکلات احتمالی وجود دارد. به این دلیل است که به سیستم‌های هم‌زمان\LTRfootnote{concurrent systems} توجه ویژه‌ای می‌شود.

%%%%%%%%%%%%%%%%%%%%%%%%%%%%%%%
\section{سیستم‌های زمان-واقعی}
سیستم‌های زمان-واقعی\LTRfootnote{real-time systems} ، سیستم‎هایی هستند که انجام عملیات و پردازش‌ها توسط آن‌ها باید در کسری از ثانیه رخ‌دهد. این سیستم‌ها محدودیت زمانی مشخصی دارند و باید آن را تضمین کنند. به این ترتیب، یک سیستم زمان-واقعی یا یک عملیات را در آن زمان معین انجام می‌دهد، یا با شکست\LTRfootnote{failure}  مواجه می‌شود.

%%%%%%%%%%%%%%%%%%%%%%%%%%%%%%%%%%%%
\section{Spin}
Spin \cite{1} محبوب ترین ابزار در جهان برای تشخیص نقص های نرم افزاری در طراحی سیستم های همزمان است. با این حال، کد های Promela را به عنوان ورودی در یافت می‌کند و نمی‌تواند برنامه های C را مستقیماً بررسی کند. بنابراین همواره سعی می‌شود تا روش‌هایی واسطه ای برای تبدیل کد C به کد Promela ابداع شود؛ تا امکان توصیف و بررسی کدهای C برای یافتن مشکلات احتمالی در برنامه های همزمان و سیستم های موازی با استفاده از Spin میسر شود.
\\
این ابزار که به صورت متن‌باز\LTRfootnote{open source}  و رایگان دردسترس همه‌ی افراد است؛ قابلیت اجرا برروی سیستم‌عامل‌های Unix ، Linux ، Mac ، Solaris و بسیاری از نسخه‌های Windows را دارد. همچنین، علاوه بر اینکه با کمک خط فرمان\LTRfootnote{command line}  قابل استفاده‌است، دارای یک رابط کاربری گرافیکی کاربرپسند\LTRfootnote{user-friendly GUI}  است، که کار با آن را راحت‌تر می‌کند. در این پژوهش، در چندین مورد، از این ابزار قوی برای وارسی و تایید\LTRfootnote{verification}  مدل‌های Promela استفاده شده‌است. به علاوه، یک راهنمای بسیار کاربردی درباره‌ی نحوه‌ی نصب و اجرای قابلیت‌های مختلف آن دردسترس ‌است.

%%%%%%%%%%%%%%%%%%%%%%%%%%%%%%%%%%%%

\section{زبان Promela}
Promela \cite{4} یک زبان مدل‌سازی فرایند است، که از آن برای تست و وارسی منطق سیستم های موازی استفاده می‌شود. برای وارسی و تایید صحت مدل‌های نوشته شده در این زبان، از ابزار Spin استفاده می‌شود.
\\
این زبان از نظر قواعد نحوی\LTRfootnote{syntactic rules} ، به زبان C شباهت‌ دارد. به همین دلیل است که می‌توان بسیاری از ساختمان‌داده‌های\LTRfootnote{data structures}  معمول و ساده‌ی موجود در زبان C را به طور مستقیم به همان نوع ساختمان‌داده‌ها در Promela ترجمه کرد. از جمله تفاوت‌های موجود در بین این دو زبان، که در این پژوهش نیز مورد بررسی قرار گرفته‌است، عدم وجود توابع \LTRfootnote{functions}  (با همان رفتار مشابه توابع در زبان C ) در زبان Promela است. بنابراین، توابع موجود در زبان C باید به رویه‌ \LTRfootnote{proctype} ها در زبان Promela ترجمه شوند و با استفاده از امکانات دیگری که در این زبان وجود دارد، عملکرد توابع زبان C به خوبی شبیه‌سازی شود. در بخش های آتی، دقیق‌تر به این موضوع پرداخته‌خواهدشد.

%%%%%%%%%%%%%%%%%%%%%%%%%%%%%
\section{زبان \lr{ (Specification and Description Language) SDL}} 
SDL یک زبان مدل‌سازی است که برای توصیف سیستم‌های زمان-واقعی استفاده می‌شود. نمودار SDL فرایند مدل‌سازی را نشان می‌دهد. این زبان می‌تواند به طور گسترده‌ در سیستم‌های خودرو، هوانوردی، ارتباطات، پزشکی و مخابرات استفاده شود.

